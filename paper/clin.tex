\documentclass[a4paper,10pt,twoside,fleqn]{article}

% Note that if you want to use the \begin{equation} ... \end{equation}
% environment, you will have to include fleqn in the
% \documentclass[...]{...} options! 
% The top of your LaTeX file should then look like this:
% \documentclass[a4paper,10pt,twoside,fleqn]{article}
\usepackage[utf8]{inputenc}
\usepackage{clin}        % Stylefile for CLIN Journal
\usepackage{harvard}     % Bibliography Stylefile
%\usepackage{...,cgloss4e,avm,trees,tree-dvips,gb4e,ipa,graphicx}
                         % Whatever other packages you need
% Harvard:
% \cite{Covington}             (Covington 1994)
% \citeasnoun{Covington}       Covington (1994)
% \citeyear{Covington}         (1994)

\pagestyle{empty}

%%% own packages
\usepackage{booktabs}
\usepackage{chngcntr}
%%% commands
\newcommand{\overbar}[1]{\mkern 2.0mu\overline{\mkern-2.0mu#1\mkern-2.0mu}\mkern 2.0mu}


\begin{document}

\title{Towards Distinctive and Typical Style Features in Authorship}


\author{Carmen Klaussner$^*$ \email{klaussnc@tcd.ie}\\
{\normalsize \bf \c{C}a\u{g}ri \c{C}öltekin}$^{**}$ \email{c.coltekin@rug.nl}\\
{\normalsize \bf John Nerbonne}$^{**}$ \email{j.nerbonne@rug.nl}
\AND \addr{$^*$Trinity College Dublin, Ireland}
\AND \addr{$^{**}$University of Groningen, The Netherlands} }


\maketitle\thispagestyle{empty} % extra pagestyle command for first page

%To be filled in by the editors
%Please leave commented out
%\jmlrheading{vol}{year}{pages}{Submission date}{Publication date}{authors}
%\copyright

\begin{abstract}
Detection of stylistic elements in authorship studies is hampered by
the lack of a gold standard that would otherwise enable us to clearly
evaluate our findings. In absence thereof, one generally resorts to choos-
ing items for which an author shows a characteristic usage compared
with other writers. In this line of work, we present both a measure for
determining characteristic elements of an author along with a method
for evaluation of those elements.
In order to select an author’s consistent features, we propose the
measure of Representativeness \& Distinctiveness \cite{prokic2012detecting} that
seeks to identify those elements that are both representative for an author
over a given set of his texts, as well distinctive with respect to an opposing
author’s sample. The method thus bears similarities with both Burrow’s
Delta \cite{Burrows2002delta} and Zeta \cite{Burrows2007all} in favouring consistent
terms that are irregular in the opposing author’s set.
Using the proposed method, we examine different types of features,
both lexical and syntactic ones, such as simple word uni-grams, but also
Part-of-Speech (POS) bi-grams/tri-grams. For evaluation, we test the
separation ability of the selected features by clustering the two authors’
documents followed by computing the Adjusted Rand Index \cite{hubert1985comparing} 
given the ideal clustering result.
We apply both feature selection and evaluation in two different
studies of authorship. In the first, we compare Charles Dickens and
Wilkie Collins, while the second one is contrasting the styles of Henry
James and Mark Twain. Testing separation ability in clustering on highly
representative and distinctive features returns results very close to the
ideal clustering result.
\end{abstract}

\section{Introduction}

Detecting stylistic features in authorship is hampered by the lack of gold
standard that would help us to evaluate our findings. 
In absence thereof, one might resort to methods that are conceived to select 
consistent features that accomodate and give preference to features that 
an author uses with a certain regularity over different works and which 
therefore hint at a clear preference. 




\subsection{The Data Sets}

Investigating two different comparisons of authors, the first set comprising Charles Dickens and 
contemporary author Wilkie Collins. 
Interesting, since Dickens' data set contains one collaboration between both authors and 
two pieces, where Dickens was main author and Dickens was among a group of colloborators. 
In terms of stylistic properties it will be illuminating to see how these behave 
with respect to similarity to the other pieces. 








\subsection{Representativeness and Distinctiveness}




Representativeness \& Distinctiveness was originally conceived in the realm of dialectometry, where it has been
shown to detect lexical items distinguishing different dialectal areas. In the context of stylometry, the method
detects elements for which an author is consistent throughout his own works while also separating him from
others. Considering, for instance, a comparison between Dickens and fellow writer Collins on word features using a
couple of novels of each writer, one first determines Dickens' representative terms, i.e. those words which he
uses consistently either frequently or infrequently over his works. In order to arrive at a combined measure, one
then favours those representative terms of Dickens that Collins uses either inconsistently or consistently but with a
different frequency over his novels. The remaining group of words are considered to be Dickens' representative
and distinctive terms when compared with Collins. Since the analysis is directional, the degree of
Representativeness of individual features being different with respect to each author, comparisons are made twice
- once from Dickens' to Collins' set and once from Collins' to Dickens' set. This returns two individual author
profiles, where features occurring in both profiles are also consistent for both writers.
Thus, Representativeness \& Distinctiveness bears similarities with both Burrow's Delta [4] and Zeta [5] in so far as
favouring consistent terms that are irregular in the opposing author's set. Additionally, it is also similar to Zeta in
being dependent on the other set for the selection of distinctive terms out of the representative ones. However,
rather than preselecting words according to different frequency strata, it is used here on the first 5000 most
frequent ones.





The technique presented here was originally applied in the domain of 

\emph{Representativeness \& Distinctiveness} \cite{prokic2012detecting} was originally applied in the realm of dialectrometry, 
a study of dialect differences 
between different sites within a language area with respect to
a choice of lexical items. The degree of difference between two sites is characterised by the aggregate differences of comparisons 
of all lexical items collected at 
each site. In the context of dialectrometry, \emph{Representativeness \& Distinctiveness} is a measure to detect characteristic 
features (lexical items), that differ little within a group of sites 
and considerably more outside that group.
Characteristic features are chosen with respect to one group $g$ of sites $|g|$ within a larger group of interest $G$, where $|G|$ 
includes the sites $s$ both within and 
outside $g$ \cite{prokic2012detecting}.

The distance function in this case is the absolute difference between the logarithm of the relative 
frequencies of $f$ with respect to two documents $d$ and $d'$.
The usual input are relative frequencies of the original term frequency weighting, which provide a 
better picture between the ratio of term frequency and document size. 
The logarithm lessens the effect of rather high frequencies. % this should be better



Thus, the distance $d_f$ between document $d$ and $d'$ with respect to feature $f$, is set as the 
absolute difference between
the logarithm of the relative frequency of their respective
input values (eq.~\ref{eq:RDinput})



\begin{equation} \label{eq:RDinput}
 d_f(d,d') = |log(relFreq(f) - log(relFreq(f')|
\end{equation}


 
$Representativeness$ of a feature $f$ for document set $D$ is then defined in eq.~\ref{eq.RDA1}


\begin{equation}\label{eq.RDA1}
 \overbar{d_f^{D}} = \frac{2}{|D|^2 -|D|} \sum\limits_{d,d' \in D,d \neq d'} d_f(d,d')
\end{equation}


The $Distinctiveness$ measure for comparing to outside documents corresponds to eq.~\ref{eq.RDA2}


\begin{equation}\label{eq.RDA2}
 \overbar{d_f^{D'}} = \frac{1}{|D|(|DS| -|D|)} \sum\limits_{d \in D, d'\notin D} d_f(d,d')
\end{equation}


 
$\overbar{d_f^{D'}}$  and $\overbar{d_f^{D}}$ are standardized by using all distance values calculated for feature $f$ to yield the 
degree of representativeness and distinctiveness for
term $dt$ in $D$ with respect to $DS$ as defined in eq.~\ref{eq.RDA3}.
 

 
\subsection{Evaluation through Clustering}

Given a list of discriminatory terms for two different author sets, we would like to ascertain to what extent the collection of terms is able to 
highlight differences between the sets and identify distinct clusters grouping the documents of different authors. 
As has been shown before, the terms used for discrimination ability should be selected according to separation ability for both author sets. 
Ideally, frequencies with respect to all terms should be consistent and fairly complementary between two author sets, e.g.~Dickens uses \emph{upon} 
consistently and frequently and Collins uses the term consistently and infrequently. 
In order to test discrimination ability of a discriminatory term list for two authors, we build a dissimilarity matrix comparing all documents in 
the complete training set. 
%dividing by the number of terms in the $dt$ list, obtaining a mean distance for all document pairs given the discriminatory terms for that iteration. 
% The present solution emphasizes differences between the two sets, since we compare for the terms with
% the highest differences between the two sets. However, comparing within sets should not be problematic, since we consider
% terms both authors either employ frequently or avoid consistently, so within one author set, we expect values for either category to be closer, either low or high.

% \subsubsection{Dissimilarity Matrix} %quote? %http://www.statistics.com/index.php?page=glossary&term_id=512
% A dissimilarity matrix (or distance matrix) $D_M$ describes pairwise distances for $M$ objects, which results in a square symmetrical $M x M$ matrix, 
% where the $ij_{th}$ entry is equal to the value of a chosen measure of distinction $d$ between the $i_{th}$ and the $j_{th}$ object.
% The diagonal elements, comparing an object to itself are not considered or are usually equal to zero.
% A sample dissimilarity matrix is shown in matrix~\ref{ex:dissimMatrix}.
% Thus, in our case each document pair in $Dickens\cup nonDickens$ is compared based on the differences of $term_i$ in a given term list. 
% A common measure of distinction $d$ would be \emph{Manhattan} or \emph{Euclidean} distance.
% 
% 
% % \begin{equation}\label{ex:dissimMatrix}
% %  D_M = \begin{pmatrix} 0           & d_{12}         & \dots                                     & d_{1j}       \cr
% %                          d_{21} &   0           &       &        \vdots             \cr
% %                            \vdots                &                 &   \ddots      &   \vdots                   \cr
% %                          d_{j1}   &     \dots                           &        & 0                 \cr \end{pmatrix}
% % \end{equation}
% 
% Clustering on the basis of dissimilarity between objects, in this case documents can be done via hierarchical clustering.
% Agglomerative hierarchical clustering, for instance is an iterative clustering process, whereby cluster objects are joined together based on a distance measure between the 
% elements within the clusters. All elements begin in their own clusters and and are joined until the desired number of output clusters has been reached. 
% A common distance measure for joining clusters together is the \emph{complete link} method, which assesses closeness on the basis of the most distant elements 
% in two clusters $X$ and $Y$, in order to avoid the merging of two clusters based on only two single elements from each set being close.
% The distance $D(X,Y)$ between clusters $X$ and $Y$ is defined in eq.~\ref{eq:completeLink}, where $X$ and $Y$ are two sets of elements or clusters
% and $d(x,y)$ is the distance between elements $x \in X$ and $y \in Y$. 
% 
% 
% \begin{equation}\label{eq:completeLink}
%  D(X,Y)= \max_{x\in X, y\in Y} d(x,y) 
% \end{equation}
%  
\subsubsection{Adjusted Rand Index for Evaluation of Clustering} 
In addition to visual clustering that gives more of an intuition of separation between two sets, a clustering result can be evaluated by 
comparing two different partitions of a finite set of objects, namely the clustering obtained and the ideal clustering. 
For this purpose, we can employ the \emph{adjusted Rand Index} \cite{Hubert1985}, which is the corrected-for-chance version of the \emph{Rand Index}. 
Given a set $S$ of $n$ elements, and two clusterings of these points, $U$ and $V$, defined as   
$U = \{ U_1, U_2, \ldots , U_r \}$ and $V = \{ V_1, V_2, \ldots , V_s \}$ with $a_i$ and $b_i$ as the number of objects in cluster $U_i$ and $V_i$ respectively.
The overlap between U and V can be summarized in a contingency table~\ref{ex:continTable}. 
where each entry $n_{ij}$ denotes the number of objects in common between $U_i\; and\; V_j: n_{ij}=|U_i \cap V_j|$.

\begin{equation} \label{ex:continTable}
 \left[n_{ij}\right]= \bordermatrix{~
U\ V &	V_1 &  V_2 &	\ldots &	V_s 	&Sums \cr
U_1 &	n_{11} &	n_{12} &	\ldots 	&n_{1s} &	a_1 \cr
U_2 &	n_{21} &	n_{22} &	\ldots &	n_{2s} &	a_2\cr
\vdots &	\vdots &	\vdots &	\ddots &	\vdots &	\vdots\cr
U_r &	n_{r1} &	n_{r2} &	\ldots &	n_{rs} &	a_r\cr
Sums &	b_1 	&b_2 	&\ldots 	&b_s &    	\cr}
 \end{equation}
 
The adjusted form of the \emph{Rand Index} is defined in eq.~\ref{eq:ARI1} and more specifically given the contingency table~\ref{ex:continTable} 
in eq.~\ref{eq:ARI}, where $n_{ij}$, $a_i$, $b_j$ are values from the contingency table. 

\begin{equation}\label{eq:ARI1}
 AdjustedIndex = \frac{Index - ExpectedIndex}{MaxIndex - ExpectedIndex}
\end{equation}

% 
% \begin{equation}\label{eq:ARI}
%  ARI = \frac{ \sum_{ij} \binom{n_{ij}}{2} - [\sum_i \binom{a_i}{2} \sum_j \binom{b_j}{2}] / \binom{n}{2} }{ \frac{1}{2} 
%  [\sum_i \binom{a_i}{2} + \sum_j \binom{b_j}{2}] - [\sum_i \binom{a_i}{2} \sum_j \binom{b_j}{2}] / \binom{n}{2}}
% \end{equation}
% 
% 
% \begin{figure}[h!]
%  \caption{Dendrogram 'complete link' of dissimilarity Matrix on the basis of 300 input terms of Dickens and Collins.}
% \label{den:dissimM}
% \includegraphics[scale=0.6]{figure/dissimEx.pdf} %0.8200016
% \end{figure}

The index is bounded between [-1,1], with 0 being the expected value and 1 the highest positive correlation between two different clusterings. 
For illustration of using the two methods presented above, we consider an example of pairwise comparison of documents 
of a dataset of  $Dickens\cup Collins$, with 55 documents belonging to Dickens and 31 to Collins. 
This yields a 86 x 86 dissimilarity matrix containing all pairwise comparisons of documents in the set.  
Figure~\ref{den:dissimM} depicts an example dendrogram showing clustering based on a dissimilarity matrix with distances computed using the  \emph{complete link} measure. 
The \emph{adjusted Rand Index} corresponding to the clustering in figure~\ref{den:dissimM} is 0.82, so very close to the ideal separation, which is also confirmed, when we consider the small number of
misclassifications (3 for Dickens and 1 for Collins).  

 


\section{Experiments}



\section{Conclusion}







%\nocite{Sag} % items in your bibliography file that are not cited in your text


\bibliographystyle{clin} 
\bibliography{biblio}  

\appendix




\section*{Dickens vs. Collins}
\begin{table}[h!] % Dickens 
\caption{Dickens' data set as part of the Dickens vs. Collins comparison.}
\label{table:Dickens-data}
\scalebox{0.8}{
\begin{tabular}{c l l l} \\\hline \hline
\textbf{No.} 	& \textbf{Author} 	& \textbf{Texts} 			& \textbf{Abbr.} \\ \hline
1   		& Dickens 		& Bleak House 				& D1023     \\
2   		& Dickens		& Great Expectations			& D1400      \\
3		& Dickens      		& Little Dorrit     			& D963       \\
4		& Dickens      		& David Copperfield     		& D766        \\
5		& Dickens      		& A Christmas Carol     		& D19337       \\
6   		& Dickens		& Life And Adventures Of Martin Chuzzlewit	& D968        \\
7		& Dickens		& The Mystery of Edwin Drood		& D564   \\
8		& Dickens      		& A Tale of Two Cities                  & D98      \\
9		& Dickens		& Master Humphrey's Clock		& D588           \\
10		& Dickens		& The Battle of Life: A Love Story      & D40723              \\
11		& Dickens		&Life And Adventures Of Nicholas Nickleby	& D967          \\  
12		& Dickens		&Barnaby Rudge      			& D917             \\
13		& Dickens		& Sketches of Young Couples		& D916        \\
14		& Dickens		& The Uncommercial Traveller            & D914           \\
15		& Dickens		& Our Mutual Friend			& D883           \\
16		& Dickens		& Pictures From Italy			& D650    \\
17		& Dickens		& Sketches by Boz			& D882      \\
18		& Dickens		& A Child's History of England       	& D699  \\
19		& Dickens		& Reprinted Pieces			& D872   \\
20		& Dickens		& Dombey and Son			& D821     \\
21		& Dickens		& Oliver Twist				& D730        \\
22		& Dickens		&The Old Curiosity Shop			& D700         \\
23		& Dickens		& American Notes			& D675         \\
24		& Dickens		&The Pickwick Papers			& D580           \\
25		& Dickens     (et al.)  & A Budget of Christmas Tales		& Dal28198         \\
26		& Dickens (et al.)	& A House to Let			& Dal2324     \\
27		& Dickens (/Collins)     & No Thoroughfare			& DC1423       \\ \bottomrule
\end{tabular}
}
\end{table}

\begin{table}[h!]% Collins 
\caption{Collins' data set as part of the Dickens vs. Collins comparison.}
\label{table:Collins-data}
\scalebox{0.8}{
\begin{tabular}{c l l l } \\\hline \hline
\textbf{No.}	& \textbf{Author} 		& \textbf{Texts} 		& \textbf{Abbr.} \\ \hline
1		& Collins       		& After Dark			& C1626    \\
2		& Collins			& Antonina			& C3606        \\
3		& Collins			& Armadale			& C1895  \\
4		& Collins			& Man and Wife			& C1586      \\
5		& Collins			& Little Novels			& C1630    \\
6		& Collins			& Jezebel's Daughter		& C3633    \\
7		& Collins			& I Say No			& C1629       \\
8		& Collins			& Hide and Seek			& C7893  \\
9		& Collins			& Basil				& C4605 \\
10		& Collins			& A Rogue's Life		& C1588     \\
11		& Collins			& The Woman in White		& C583         \\
12		& Collins			& The Two Destinies		& C1624    \\
13		& Collins			& The Queen of Hearts		& C1917        \\
14		& Collins			& The New Magdalen		& C1623     \\
15		& Collins			& The Moonstone			& C155         \\
16		& Collins			& The Legacy of Cain		& C1975       \\
17		& Collins			& The Law and the Lady		& C1622         \\
18		& Collins			& The Haunted Hotel: A Mystery of Modern Venice	&    C170              \\
19		& Collins			& The Fallen Leaves		& C7894           \\
20		& Collins			& The Evil Genius		& C1627           \\
21		& Collins			& No Name			& C1438        \\
22		& Collins			& Poor Miss Finch		& C3632           \\
23		& Collins			& Rambles Beyond Railways	& C28367         \\
24		& Collins			& The Black Robe		& C1587      \\
25		& Collins			& Miss or Mrs.?			& C1621          \\
26		& Collins			& My Lady's Money		& C1628         \\
27		& Collins			& The Dead Alive		& C7891        \\
\bottomrule
\end{tabular}
}
 \end{table}




\end{document}
