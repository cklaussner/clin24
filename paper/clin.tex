\documentclass[a4paper,10pt,twoside,fleqn]{article}

% Note that if you want to use the \begin{equation} ... \end{equation}
% environment, you will have to include fleqn in the
% \documentclass[...]{...} options! 
% The top of your LaTeX file should then look like this:
% \documentclass[a4paper,10pt,twoside,fleqn]{article}
\usepackage[utf8]{inputenc}
\usepackage{clin}        % Stylefile for CLIN Journal
\usepackage{harvard}     % Bibliography Stylefile
%\usepackage{...,cgloss4e,avm,trees,tree-dvips,gb4e,ipa,graphicx}
                         % Whatever other packages you need
% Harvard:
% \cite{Covington}             (Covington 1994)
% \citeasnoun{Covington}       Covington (1994)
% \citeyear{Covington}         (1994)

\pagestyle{empty}

%%% own packages
\usepackage{booktabs}
\usepackage{chngcntr}
\usepackage{rotating}
\usepackage[section]{placeins}
\usepackage[super]{nth}
\usepackage{wrapfig}
\usepackage{subfig}
\usepackage{chronology}
%%% commands
\newcommand{\overbar}[1]{\mkern 2.0mu\overline{\mkern-2.0mu#1\mkern-2.0mu}\mkern 2.0mu}


\begin{document}

\title{Towards Distinctive and Typical Style Features in Authorship}


\author{Carmen Klaussner$^*$ \email{klaussnc@tcd.ie}\\
{\normalsize \bf \c{C}a\u{g}ri \c{C}öltekin}$^{**}$ \email{c.coltekin@rug.nl}\\
{\normalsize \bf John Nerbonne}$^{**}$ \email{j.nerbonne@rug.nl}
\AND \addr{$^*$Trinity College Dublin, Ireland}
\AND \addr{$^{**}$University of Groningen, The Netherlands} }


\maketitle\thispagestyle{empty} % extra pagestyle command for first page

%To be filled in by the editors
%Please leave commented out
%\jmlrheading{vol}{year}{pages}{Submission date}{Publication date}{authors}
%\copyright

\begin{abstract}
Detection of stylistic elements in authorship studies is hampered
by the lack of a gold standard that would otherwise enable us to 
clearly evaluate our findings. In absence thereof, one generally 
resorts to choosing items for which an author shows a 
characteristic usage compared with other writers. 
In this line of work, we present both a measure for determining 
characteristic elements of an author that he uses consistently
over different works by examining different types of features, 
both lexical and syntactic ones.
For evaluation, we test the separation ability of the selected 
features by clustering the data set on their basis.

We apply both feature selection and evaluation in two different
studies of authorship. In the first, we compare Charles Dickens 
and Wilkie Collins, while the second one is contrasting the 
styles of Henry James and Mark Twain. Testing separation ability 
in clustering on highly representative and distinctive features 
returns high results for all feature types in Dickens 
vs. Collins and perfect clustering results for all feature types
in the Twain/James comparison.
\end{abstract}




\section{Introduction}

The concept of \emph{style} in studies of authorship refers to 
something like the \emph{feel} of a piece of text, which might be less
tangible than style in other disciplines, such as art or music, 
where we use \emph{language} to express aspects of a 
painting or a piece of music, whereas for \emph{style} in writing, 
we are left with the same tools for explaining something that were
used in creating it. 

While traditional studies of authorship involve individual literary
scholars judging upon what constitutes the style of an author, 
non-traditional studies employ statistical techniques to discover 
the style of an author automatically. 
In either scenario, the development and investigation of suitable
methods of detection is hampered by the lack of a gold standard,
that would be able to indicate the method's closeness to 
revealing true stylistic elements of an author. 
In absence thereof, the weight lies primarily on the method's 
theoretical appropriateness in the way it selects stylistic 
markers, which results in the need to understand the methods 
we employ as well as their implications. 

% this part needs to include more references
\emph{Consistency} of stylistic markers has been somewhat of 
a central theme to studies of authorship, such as in early 
studies \cite{Mosteller2008}, which also emphasised  the need
to accumulate a large number of markers to increase reliability 
of the result.
This has been continued in more recent studies, as in 
the development of the Burrow's Delta \cite{Burrows2002delta} 
and Zeta \cite{Burrows2007all} measures that build on
features representative for an author over a given number
of his texts, while distinguishing him from another/other 
author(s). 

%%% elaborate:
%% Burrows' methods are primarily for authorship attribution,i.e.
% determine authorship of particular document rather than
% stylistic features/authorship fingerprint 

In this work, we consider \emph{Representativeness \& Distinctiveness}
\cite{prokic2012detecting}, a technique originating in dialectometry,
to select an author’s consistent features within his own texts, that
are simultaneously also distinctive with respect to another author’s texts.
Further, we propose a heuristic method for evaluation of 
selected features that is intended to measure how well they are able to
separate the data set into the correct groupings. 
As part of the analysis, we include both lexical and syntactic features, 
such as plain word uni-grams, but also Part-of-Speech (POS) 
bi-grams/tri-grams.
The data consists of two separate authorship sets, where the first
contains texts by two British authors, Charles Dickens and
Wilkie Collins and the second comprises writings by the two
American authors, Henry James and Mark Twain. 
Related work in stylometry includes a study of Dickens' style using
Random Forests classification and
comparing both Collins and a larger reference set comprising
texts from the \nth{18}  and \nth{19} century \cite{Tabata2012}.
Further, there has been a corpus stylistics study \cite{mahlberg2007clusters},
aiming to extract Dickens-specific key word clusters (sequences of words), 
that can be interpreted as pointers to more general functions, such
as \emph{Body Part} clusters pertaining to Dickens' particular affinity for using 
body parts for individualisation of characters, e.g. ``his hands in his pockets''. 
Recently, Henry James was closely analysed for style change over time by
applying Burrow's methods to his early and late works separately, which 
when analysed together formed distinct clusters \cite{Hoover2012}. 


Thus, in section~\ref{sec:data}, we describe the two data sets and then continue with 
introducing \emph{Representativeness \& Distinctiveness} 
in section~\ref{sec:method}. Further, section~\ref{sec:evaluation}
introduces and illustrates our proposed method of evaluation. 
Finally, in section~\ref{sec:experiments}, we describe our experiments
and results with respect to the two data sets and section~\ref{sec:conclusion}
closes the discussion. 


\section{The Data Sets} \label{sec:data}
For this study, we built two different authorship data sets, one for comparing Charles Dickens
and contemporary writer Wilkie Collins and the other one opposing Henry James 
to fellow American writer Mark Twain. 
All data was obtained from \emph{Project Gutenberg}\footnote{http://www.gutenberg.org/}
and the \emph{Internet Archive}\footnote{https://archive.org/}.

\subsection{Dickens vs. Collins}
% connection Dickens/Collins
Table~\ref{table:Dickens-data} and table~\ref{table:Collins-data} show Dickens'
and Collins' data set respectively, both comprising 27 documents.
In addition to texts solely authored by himself, Dickens' data set also contains
three books, namely \emph{A Budget of Christmas Tales}, \emph{A House to Let} and \emph{No Thoroughfare}
that were collaborations, but for which Dickens seems to have been main author.
In particular, the last two also include Wilkie Collins as co-author and notably the work
\emph{No Thoroughfare} was written only by Dickens and Collins. 
The collaborations were included to provide some interesting insight with respect to stylistic properties.
If both authors persist in terms of style, these works should be somewhat more difficult 
to classify than those written by only one of them.  
 

\subsection{James vs. Twain}
While Dickens and Collins seemed to have been close enough to collaborate on work,
this might have been a highly unlikely scenario for Henry James and Mark Twain. 
Despite both being American writers very close in age, they did not seemed to have
particularly approved of each other as artists \cite{canby1951turn}. 
For this reason it might be interesting to investigate to what extent this mutual dislike and 
disapprobation of each other 's work takes manifestation in their writings
and what elements can be used best to separate them. 
Table~\ref{table:Twain-data} and table~\ref{table:Twain-data} show 
James' and Twain's data sets, where James is represented with 25 and Twain
with 21 works. 




% Dickens vs. Collins
\begin{table}[!ht]
\centering
\caption{Dickens' data set.}
      \label{table:Dickens-data}
      \small
\begin{tabular}{c l l l} \\\hline \hline
\textbf{No.} 	& \textbf{Author} 	& \textbf{Texts} 			& \textbf{Abbr.} \\ \hline
1   		& Dickens 		& Bleak House 				& D1023     \\
2   		& Dickens		& Great Expectations			& D1400      \\
3		& Dickens      		& Little Dorrit     			& D963       \\
4		& Dickens      		& David Copperfield     		& D766        \\
5		& Dickens      		& A Christmas Carol     		& D19337       \\
6   		& Dickens		& Life And Adventures Of Martin Chuzzlewit	& D968        \\
7		& Dickens		& The Mystery of Edwin Drood		& D564   \\
8		& Dickens      		& A Tale of Two Cities                  & D98      \\
9		& Dickens		& Master Humphrey's Clock		& D588           \\
10		& Dickens		& The Battle of Life: A Love Story      & D40723              \\
11		& Dickens		& Life And Adventures Of Nicholas Nickleby	& D967          \\  
12		& Dickens		& Barnaby Rudge      			& D917             \\
13		& Dickens		& Sketches of Young Couples		& D916        \\
14		& Dickens		& The Uncommercial Traveller            & D914           \\
15		& Dickens		& Our Mutual Friend			& D883           \\
16		& Dickens		& Pictures From Italy			& D650    \\
17		& Dickens		& Sketches by Boz			& D882      \\
18		& Dickens		& A Child's History of England       	& D699  \\
19		& Dickens		& Reprinted Pieces			& D872   \\
20		& Dickens		& Dombey and Son			& D821     \\
21		& Dickens		& Oliver Twist				& D730        \\
22		& Dickens		&The Old Curiosity Shop			& D700         \\
23		& Dickens		& American Notes			& D675         \\
24		& Dickens		&The Pickwick Papers			& D580           \\
25		& Dickens    (et al.)   & A Budget of Christmas Tales		& Dal28198         \\
26		& Dickens (et al.)	& A House to Let			& Dal2324     \\
27		& Dickens (/Collins)     & No Thoroughfare			& DC1423       \\ \bottomrule
\end{tabular}
\end{table}
\begin{table}[!ht]
\centering
\caption{Collins' data set.}
\label{table:Collins-data}
\small
\begin{tabular}{c l l l } \\\hline \hline
\textbf{No.}	& \textbf{Author} 		& \textbf{Texts} 		& \textbf{Abbr.} \\ \hline
1		& Collins       		& After Dark			& C1626    \\
2		& Collins			& Antonina			& C3606        \\
3		& Collins			& Armadale			& C1895  \\
4		& Collins			& Man and Wife			& C1586      \\
5		& Collins			& Little Novels			& C1630    \\
6		& Collins			& Jezebel's Daughter		& C3633    \\
7		& Collins			& I Say No			& C1629       \\
8		& Collins			& Hide and Seek			& C7893  \\
9		& Collins			& Basil				& C4605 \\
10		& Collins			& A Rogue's Life		& C1588     \\
11		& Collins			& The Woman in White		& C583         \\
12		& Collins			& The Two Destinies		& C1624    \\
13		& Collins			& The Queen of Hearts		& C1917        \\
14		& Collins			& The New Magdalen		& C1623     \\
15		& Collins			& The Moonstone			& C155         \\
16		& Collins			& The Legacy of Cain		& C1975       \\
17		& Collins			& The Law and the Lady		& C1622         \\
18		& Collins			& The Haunted Hotel: A Mystery of Modern Venice	&    C170              \\
19		& Collins			& The Fallen Leaves		& C7894           \\
20		& Collins			& The Evil Genius		& C1627           \\
21		& Collins			& No Name			& C1438        \\
22		& Collins			& Poor Miss Finch		& C3632           \\
23		& Collins			& Rambles Beyond Railways	& C28367         \\
24		& Collins			& The Black Robe		& C1587      \\
25		& Collins			& Miss or Mrs.?			& C1621          \\
26		& Collins			& My Lady's Money		& C1628         \\
27		& Collins			& The Dead Alive		& C7891        \\
\bottomrule
\end{tabular}  
\end{table}




%James vs. Twain
\begin{table}[!htb]
    %\caption{Twain' and James' data set as part of the Twain vs. James' comparison.}
   % \label{table:TJ}
    \begin{minipage}{.62\linewidth}
    \centering
      \caption{Twain's data set.} %necessary???
      
      \label{table:Twain-data}
\begin{tabular}{c l l l} \\\hline \hline
\textbf{No.} 	& \textbf{Author} 	& \textbf{Texts} 			& \textbf{Abbr.} \\ \hline
1		&Twain			&Innocents Abroad			& T-ia\\ 
2		&Twain			&The Gilded Age: A Tale of Today	& T-tgaatot\\ 
3		&Twain			&Sketches New and Old			& T-snao\\
4		&Twain			&The Adventures of Tom Sawyer		& T-taots\\ 
5		&Twain			&A Tramp Abroad				& T-ata\\ 
6		&Twain			&Roughing It				& T-ri \\ 
7		&Twain			&The Prince and the Pauper		& T-tpatp\\ 
8		&Twain			&Life on the Mississippi		& T-lotm\\ 
9		&Twain			&The Adventures of Huckleberry Finn	& T-taohf\\ 
10		&Twain			&A Connecticut Yankee in King Arthur's Court& T-acyikac\\  
11		&Twain			&The American Claimant			& T-tac\\  
12		&Twain			&The Tragedy of Pudd'nhead Wilson	&T-ttopw\\   
13		&Twain			&Tom Sawyer Abroad			& T-tsa\\ 
14		&Twain			&Tom Sawyer Detective			& T-tsd\\ 
15		&Twain			&Personal Recollections of Joan Arc	& T-proja\\ 
16		&Twain			&Following the Equator: A Journey Around the World	&T-fteajatw\\
17		&Twain			&Those Extraordinary Twins 		&T-tet\\ 
18		&Twain			&A Double Barrelled Detective Story	& T-adbds\\ 
19		&Twain			&Christian Science			& T-cs\\
20		&Twain			&Chapters from My Autobiography		& T-cfma \\ 
21		&Twain			&The Mysterious Stranger		& T-tms\\ 
\bottomrule
\end{tabular}
\end{minipage}%
\vfill
    \begin{minipage}{.62\linewidth}
      \centering
      \caption{James' data set.} %necessary???
\label{table:James-data}
\begin{tabular}{c l l l } \\\hline \hline
\textbf{No.}	& \textbf{Author} 		& \textbf{Texts} 		& \textbf{Abbr.} \\ \hline
1		&James				&The American			&J-ta	\\
2		&James				&Watch and Ward			&J-waw	\\
3		&James				&The Europeans			&J-te	\\ 
4		&James				&Confidence 			&J-c	\\
5		&James				&Washington Square 		&J-ws	\\ 
6		&James				&Portrait of a Lady 		&J-poal	\\ 
7		&James				&Roderick Hudson		&J-rh	 \\
8		&James				&The Bostonians 		&J-tb	\\
9		&James				&Princess Casamassima 		&J-pc	\\
10		&James				&The Reverberator 		&J-tr	\\ 
11		&James				&The Aspern Papers		&J-tap	\\ %  novella
12		&James				&The Tragic Muse 		&J-ttm	\\ %
13		&James				&The Other House 		&J-toh	\\ %
14		&James				&What Maisie Knew        	&J-wmk	\\ 
15		&James				&The Spoils of Poynton  	&J-tsop	\\ %
16		&James				&Turn of the Screw		&J-tots	\\ %novella
17		&James				&The Awkward Age 		&J-taa	\\ %
18		&James				&The Sacred Fount		&J-tsf	\\ 
19		&James				&The Wings of the Dove		&J-twotd	\\  
20		&James				&The Golden Bowl 		&J-tgb	\\ %
21		&James				&The Ambassadors		&J-tamb	\\ 
22		&James				&The Outcry			&J-to	\\ %
23		&James				&The Ivory Tower (unfinished)	&J-tit	\\ %
24		&James				&The Sense of the Past (unfinished)&J-tsotp	\\%
25		&James				&In the Cage 			&J-itc	\\ % novella
\bottomrule
\end{tabular}

    \end{minipage} 
\end{table}



\section{Representativeness and Distinctiveness for Stylometry} \label{sec:method}
The statistical technique of Representativeness and Distinctiveness
originated in the realm of dialectometry, where it has been
shown to detect lexical items able to distinguish different dialectical 
areas \cite{prokic2012detecting}.

In the context of stylometry, the method can be employed to detect elements for 
which an author is consistent throughout his own works while also separating him 
from another author. 
Considering, for instance, a comparison between Dickens and fellow 
writer Collins on word features using a couple of novels of each writer, 
one first determines Dickens' representative terms, i.e. those words which he
uses consistently either frequently or infrequently over his works. 
In order to arrive at a combined measure, one then favours those representative 
terms of Dickens that Collins uses either inconsistently or consistently but with a
different frequency over his novels. The remaining group of words are considered 
to be Dickens' representative and distinctive terms when compared to Collins. 

Thus, Representativeness and Distinctiveness bears similarities with both 
Burrow's Delta \cite{Burrows2002delta} and Zeta  \cite{Burrows2007all} in
favouring consistent terms that are irregular in the opposing author's set. 
Additionally, it is also similar to Zeta in being dependent on the other set 
for the selection of distinctive terms out of the representative ones. 
However, the technique can be applied to all frequency strata simultaneously.
While Burrow's methods are primarily intended for the actual attribution
of a document, Representativeness and Distinctiveness would be appropriate
for detecting items distinguishing two authors as well as ones that 
the authors would either consistently avoid or favour with respect to 
the other author. 
Formally the method of Representativeness and Distinctiveness is defined 
as follows: \\

\textsc{Representativeness} of a feature $f$ for document set $D$ is defined 
in eq.~\ref{eq.Rep}. \\


 \begin{equation}\label{eq.Rep}
 \overbar{d_f^{D}} = \frac{2}{|D|^2 -|D|} \sum\limits_{d,d' \in D,d \neq d'} d_f(d,d')
 \end{equation}
 
The \textsc{Distinctiveness} measure for comparing to outside documents corresponds 
to eq.~\ref{eq.Dis}.\\


\begin{equation}\label{eq.Dis}
 \overbar{d_f^{D'}} = \frac{1}{|D|(|DS| -|D|)} \sum\limits_{d \in D, d'\notin D} d_f(d,d')
\end{equation}
The distance $d_f$ between document $d$ and $d'$ with respect to feature $f$, is set as the 
absolute difference between the logarithm of the relative frequency of their respective
original term frequencies (eq.~\ref{eq.dist}).
Relative frequency is given preference here to normalise over document size, while 
taking the logarithm lessens the effect of rather high frequencies.

\begin{equation} \label{eq.dist}
 d_f(d,d') = |log(relFreq(f) - log(relFreq(f')|
\end{equation}
$\overbar{d_f^{D'}}$  and $\overbar{d_f^{D}}$ are standardised by using all 
distance values calculated for feature $f$ to yield the 
degree of representativeness and distinctiveness for
feature $f$ in $D$ with respect to $DS$ as defined in eq.~\ref{eq.comb}.
 
 \begin{equation}\label{eq.comb}
\frac{\overbar{d^{D'}_f} - \overbar{d_f} }{sd(d_f)} - \frac{\overbar{d^{D}_f} - \overbar{d_f}}{sd(d_f)}
\end{equation}


\subsection{Determining Stylistic Author Profiles}

When applying Representativeness \& Distinctiveness for building author profiles, 
one needs to take into consideration that there is an ambiguity connected to 
a term being  distinctive from one author to the other. 
We consider the previous case, where we are trying to find features that 
separate Dickens and Collins. 
As a first step, we identify Collins' representative terms, which is only based on
comparing his own documents for different features. Once, this is completed, 
we identify those terms that separate Collins' set well from Dickens' set
of documents. 
Finally, we choose the highest representative ones that are also
discriminative with respect to Dickens. 

\begin{wrapfigure}{r}{0.4\textwidth}
  \caption{Representativeness}
 \begin{center}
  \includegraphics[width=0.4\textwidth]{figures/repres1-fin.png}
  \end{center}
 \end{wrapfigure}
 
However, this analysis is directional in the sense that the degree
of representativeness of a certain feature could be different 
for the opposing set, i.e. there are two different scenarios
for a feature being \emph{different} in the opposing set. 

Case 1. The term $t_i$ is consistent in Collins' set $C$ with a low
frequency, while the same term $t_i$ is consistent in Dickens' set $D$
with a high frequency. 
Thus, the term is representative and distinctive for both sets, 
even though we did not consider the \emph{Representativeness} for set $D$. 
Obviously, the converse could also be true: a consistently high frequency 
of term $t_i$ for set $C$ and a consistently low frequency for the set $D$.
This first case does not create any issues for measuring similarity over the
complete data set, since documents of one author set will be similar on 
these features while at the same time being different to the respective other set.



\begin{wrapfigure}{r}{0.4\textwidth}
\caption{\textsc{Distinctiveness}}
{\scriptsize \textsc{Case 1}}\\
  \includegraphics[scale=0.2,width=\linewidth]{figures/distinc1-fin.png}
{\scriptsize \textsc{Case 2}}\\
  \includegraphics[scale=0.2,width=\linewidth]{figures/distinc2-fin.png}
\end{wrapfigure}


Case 2. The second possibility is the one that may cause issues.
Assuming a representative and distinctive term for set $C$, distributed with a 
frequency either high or low. 
However, in this case the same term is not representative for set $D$ and values 
may fluctuate from high to low. Although this term is not representative 
for $D$, it is distinctive from $C$ to $D$, because it is constant in $C$ 
while not being so in $D$. 
Clustering the data set on the basis of terms such as these may create noise, 
since it will not show similarities for documents within $D$ and 
may have occasional rather similar values to the ones in $C$ that would
then rate it closer to documents in $C$.

In order to identify elements representative and distinctive for both sets,
the analysis has to be carried out for both sets and having identified
two separate author profiles, we select those features that are shared
by both profiles.

\subsection{Evaluation of Distinctive Markers} \label{sec:evaluation}
Evaluation in studies of authorship can only be done heuristically
by identifying desirable properties of stylistic markers and
determining means of measuring this property. 
The stylistic markers' ability to separate the authorship set in 
question could be considered as one of these desirable characteristics. 
For instance, having identified a set of discriminatory markers
for Dickens and Collins using a particular method, we evaluate 
that method by testing whether those markers are indeed able to 
cluster the document space appropriately into the two groups of
documents. 
Thus, discrimination ability of a set of distinctive markers
is determined by the overall similarity grouping of documents 
according to those markers. 
Poor discriminators will result in poor clustering, where
authors are not separated well into two groups, while 
good discriminators should be able to divide documents of 
different origin clearly into two different groups. 
This can be quantified by evaluating the clustering result
using the \emph{Adjusted Rand Index} \cite{hubert1985comparing},
which compares the current clustering result to the ideal 
result by pairwise comparison of the groups. 
The returned index is bounded by [-1,1], with 0 being the 
expected value and 1 the highest positive correlation 
between two different clustering versions. 
%%%TODO complete Link explanation

\section{Experiments}\label{sec:experiments}
As part of the experiments, we are looking at both data sets
for different feature types, in this case word uni-grams,
POS bi-grams and tri-grams. 
The best features for each individual author are selected by 
computing a threshold based on the representative and distinctive
scores of all features within. 
Thus, for an authors' full list of features, one first computes 
the mean $\mu_f$ over all representative and distinctive values 
for all features and multiplies this by a fixed value, e.g. 1.8
that is dependent on the feature input size and individual
requirements. In this case, only features with a score higher than
$\mu_f \times 1.8$ would be kept. 

In section~\ref{sec:dickens-collins}, we look at distinctive
features of Dickens compared to Collins and section~\ref{sec:twain-james}
opposes James and Twain in style analysis. 

\subsection{Dickens vs. Collins} \label{sec:dickens-collins}
Both Dickens' and Collins' data set individually contained 27 novels,
where three in Dickens' set were collaborations with other authors and 
one only with Collins. The uni-gram input matrix was a 
54-documents x 5000 features matrix, where the features were 
the 5000 most frequent features. 
Similarly, for POS tri-grams, there was a 54 x 5000 input matrix and
for POS bi-grams, it was 54 x 1100. 
The author profiles for word uni-grams and POS tri-grams were chosen
by using a threshold of $\mu_f \times 1.9$ and for
POS bi-grams by using $\mu_f \times 1.7$. 



\begin{table}[h!]
\small

\caption{Dickens' and Collins' highest rated features }
\label{tab:results-features-dc}
\begin{tabular}{lllllll}\toprule[1.2pt]
 \multicolumn{1}{c}{\textbf{No}} & \multicolumn{2}{c}{\textbf{Word uni-grams}}&  
 \multicolumn{2}{c}{\textbf{POS bi-grams}} & \multicolumn{2}{c}{\textbf{POS trigrams}}\\
 \cmidrule(r){1-1} 
 \cmidrule(r){2-3} 
\cmidrule(r){4-5} 
 \cmidrule(r){6-7}
    & Dickens	  &  Collins    & Dickens&  Collins& Dickens	  &  Collins   \\\midrule
1   &       upon  &       upon  &  RP.CC  &   NN.TO  &  CC.NN.CC   & IN.VBG.PP \\
2   &   scarcely  & discovered  &  IN.IN  &   RB.TO  &  NN.CC.RB   &   TO.DT.NN  \\
3   & discovered  &   produced  &  NN.TO  &   RP.CC  &  RB.IN.IN   &  IN.VBG.DT \\
4   &       many  &  interests  & VBG.RP  &  TO.PP  & CC.VBG.RP   &  PP..JJ.JJ \\
5   &        and  &       left  &  CC.RB  & VBP.VBD  & VBN.RP.CC   &   DT.NN.TO \\
6   &       left  &       many  & CC.VBG  &   NN.NN  &  RB.MD.VB   & VBD.IN.VBG \\
7   &       very  &    useless  &  JJ.CC  &  NNS.NN  & RB.TO.PP   & VBN.IN.VBG \\
8   &        but  &    attempt  &  NN.CC  &   RB.NN  & CC.VBG.IN   &  DT.NN.PP \\
9   &       much  &     motive  &  CC.IN  &   NP.NN  & DT.NN.PP   &   NN.NN.NN \\
10  &     beside  &       only  &  DT.CC  &   RB.JJ  &  RB.JJ.CC   & NN.WDT.PP \\
11  &     though  &       risk  &  RB.CC  & VBZ.VBD  &  JJ.NN.CC   &  TO.PP.NN \\
12  &       down  & resolution  &  RB.TO  &  CC.WRB  &  IN.IN.DT   &  TO.VB.PP \\
13  &   produced  &      words  & VBG.JJ  &  CC.VBG  & CC.VBG.CC   & VBD.TO.PP \\
14  &    several  &  hesitated  &  IN.CC  &  VBG.RP  &  RB.RB.CC   &  RB.TO.PP \\
15  &       lest  &       very  &  CC.NN  & VBN.PP  & VBG.RP.CC   & PP..NNS.PP \\
16  &     failed  &   scarcely  & CC.WRB  & NNS.VBP  & NNS.CC.RB   & VBD.NP.NNS \\
17  &    control  &     future  & VBG.CC  &  VBN.TO  &  DT.NN.TO   & PP.NN.POS \\
18  &      great  &     return  & RP.NNS  &   RB.MD  &  CC.JJ.CC   &   PP.DT.NN \\
19  &       such  &      first  & PDT.DT  &  NP.NNS  & RB.CC.VBG   &   DT.NN.PP \\
20  &     indeed  &     failed  &  CC.JJ  &  VB.PP  &  CC.RB.RB   &  VBD.TO.DT \\
\bottomrule
\end{tabular}
\end{table}
Table~\ref{tab:results-features-dc} shows the highest rated features 
according to Representativeness and Distinctiveness that are best 
in discriminating between the two authors. The values for Dickens' features 
were calculated by comparing to Collins and vice versa. 
Column two and three show the top 20 uni-gram features for the two authors
separately, while column four and five show POS bi-grams. 
For the word uni-grams, 8 of the 20 terms appear in both lists, so this corresponds
to Case 1 of Distinctiveness, where they are consistent for both authors, but
have very different frequency distributions, here \textsc{upon, many, left, discovered}
appear among the top 6 in both lists. 
For the POS bi-grams, one can observe that  6 features, namely
\textsc{NN.TO, RB.TO, RP.CC, CC.WRB, CC.VBG, VBG.RP} 
are shared by both authors over the first 20 highest items. 
Noticeable is the frequent appearance of the CC and VBG tag, 
corresponding to `coordinating conjunction' and `Verb in gerund or
past particle' respectively. 
For the POS tri-grams, the list of shared items decreases and only three remain 
common to both authors for the first 20: \textsc{DT.NN.TO,  DT.NN.PP, RB.TO.PP}.
Notable here are sequences involving possessive pronoun (PP) and TO (to) after 
a noun phrase. 
%%% Natural language example!!!

The fact that more items are shared for word uni-grams is reflected in the evaluation using 
clustering, shown in table~\ref{tab:results-clust}.
For clustering we use the features shared by both authors by intersecting both profiles. 
The validation method used to create individual subsets of the total data set 
is \emph{Leave-one-out} cross-validation. 
The results for the Adjusted Rand Index for word uni-grams are consistently high with 
almost all values being around 93\%. In comparison, POS bi-grams fare less well
with an average of ...
%%% TODO : mean /std error
POS tri-grams are slightly better, with an average of ...
This might be explained by considering that POS bi-grams are a more general 
construction than POS tri-grams (there are also less of them) and they might
be worse in discriminating between authors of similar backgrounds 
than tri-grams or even word uni-grams. 


Figure~\ref{fig:dc-unigram-mdsplot} and figure~\ref{fig:dc-unigram-dendo} 
show the MDS plot and dendrogram corresponding to the shared features of 
iteration 27.
The dendrogram is computed using the ``complete link'' similarity 
measure.\footnote{All MDS plots and dendrograms 
were created using Gabmap \cite{nerbonne2011gabmap}: http://www.gabmap.nl/. }
The two author groups seem to form roughly to clusters. 
Interestingly, the combined Dickens-Collins piece, \emph{DC1423} is 
(mis)classified as one of Collins' set. One of the other collaborations 
\emph{Dal2324} is also approaching the border, although from the correct
side. The other collaboration, \emph{Dal28198} lies at the centre of
Dickens' cluster. 
Another Collins' piece, \emph{C28367} corresponding to 
\emph{Rambles Beyond Railways} that appears dangerously close to 
unfriendly territory has also been noted as suspicious in the 
previous study of Dickens/Collins \cite{Tabata2012}.
The dendrogram confirms the overall analysis, showing two distinct 
clusters, with \emph{Dal28198} and \emph{DC1423} being allocated
to Collins' cluster. 



\begin{figure}
\centering
\parbox{8cm}{
\includegraphics[width=8cm,height=4.5cm]{mdsplot/dc-wuni-DC1423.pdf}
\caption{MDS plot on Dickens and Collins highest shared word uni-grams 
corresponding to iteration 27.}
\label{fig:dc-unigram-mdsplot}}
\qquad
\begin{minipage}{6cm}
\includegraphics[width=6cm]{dend/dc-wuni-cl-DC1423.pdf}
\caption{Dendogram on Dickens and Collins highest shared word uni-grams 
corresponding to iteration 27}
\label{fig:dc-unigram-dendo}
\end{minipage}
\end{figure}


\subsection{James vs. Twain} \label{sec:twain-james}
The James vs. Twain comparison contained 46 documents, where 
25 belonged to Henry James and 21 to  Mark Twain. 
As before, we tested three different feature types, word uni-grams, 
POS bi-grams and POS tri-grams. 
For POS bi-grams, there was an 46 x 1100 input matrix and for 
the two other feature types, POS tri-grams and word uni-grams, 
there was an 46 x 5000 input matrix. 
The author profiles for POS tri-grams were chosen
by using a threshold of $\mu_f \times 1.9$ and for
POS bi-grams and word uni-grams a threshold of $\mu_f \times 1.7$
was used. 
%(dtm.wuni,5000 1.7)
%(posbi 1100 1.7) 
%(dtm.postri,5000,1.9)

\begin{table}[h!]
\small

\caption{James' and Twain's highest rated features }
\label{tab:results-features-tj}
\begin{tabular}{lllllll}\toprule[1.2pt]
 \multicolumn{1}{c}{\textbf{No}} & \multicolumn{2}{c}{\textbf{Word unigrams}}&  
 \multicolumn{2}{c}{\textbf{POS bigrams}} & \multicolumn{2}{c}{\textbf{POS trigrams}}\\
 \cmidrule(r){1-1} 
 \cmidrule(r){2-3} 
\cmidrule(r){4-5} 
 \cmidrule(r){6-7}
    & James	  &  Twain    &   James&  Twain& James	  &  Twain  \\\midrule
1   &  companion  &  companion  &   NN.CC  &  NN.CC  &   DT.NN.CC   &  CC.VB.DT\\
2   &     around  &     around  &   CC.VB  &  CC.VB  &  NN.IN.WDT   &  DT.NN.CC\\
3   &        and  &        and  &   CC.DT  &  CC.DT  &   CC.VB.DT   &  NN.CC.VB\\
4   &  conscious  &      along  &  PP..RB  & IN.WDT  &  CC.VBD.RB   & NNS.CC.RB\\
5   &    million  &       body  &   CC.JJ  & NNS.CC  &  IN.WDT.PP   & NNS.CC.VB\\
6   &      spite  &      might  &  IN.WDT  &  RB.TO  &  NNS.RB.CC   & VB.DT.NNS\\
7   &        two  &   declared  &   RB.CC  &  CC.NN  &   TO.PP.IN   & NNS.RB.CC\\
8   &      might  &  everybody  &  NN.RBR  & CC.VBD  &   NN.RB.CC   &  NN.RB.CC\\
9   &      maybe  &      spite  &   CC.CD  &  CC.JJ  &  VBN.TO.PP   &  CC.VB.IN\\
10  &     rather  &    chapter  &  TO.WDT  & PP..RB  &  NN.TO.WDT   & CC.VBD.DT\\
11  &    pretend  &  conscious  &   RP.DT  &  VB.DT  &   NN.CC.VB   & NN.CC.NNS\\
12  &      least  &      least  &  CC.VBD  & DT.NNS  & IN.WDT.PP.   & CC.DT.NNS\\
13  &  everybody  &   charming  &  DT.NNS  & CC.NNS  &  NNS.CC.RB   & IN.WDT.PP\\
14  &      along  &        two  &  NNS.CC  & TO.WDT  &   TO.VB.PP   & DT.NNS.CC\\
15  &      drunk  &     having  &  RBR.RB  &  CC.RB  &  PP..RB.JJ   & NN.IN.WDT\\
16  &     simply  &   children  &  NPS.NP  & RBR.RB  &  DT.NNS.CC   &  RB.TO.VB\\
17  & companions  &    thunder  & WDT.PP.  & CC.VBP  &  RB.VBN.PP   &  TO.PP.IN\\
18  &        her  &     twelve  &   WP.PP  &  RB.CC  &  RB.DT.NNS   & RB.DT.NNS\\
19  &      sense  &   crippled  &  VBN.PP  &  PP.IN  &  NNS.CC.VB   & RB.CC.VBD\\
20  &   declared  & procession  &  RBR.JJ  &  CC.CD  &  CC.DT.NNS   &  CC.RB.DT\\

\bottomrule
\end{tabular}
\end{table}

Table~\ref{tab:results-features-tj} shows the  20 highest rated 
(in terms of Representativeness/Distinctiveness) 
features for all three feature types. Each list of top 
features is distinctive with respect to the
other author in the comparison. 
A closer look at common features of Twain and James shows that
they share 11 out of 20 uni-grams: 
\textsc{companion, around, and, along, might, declared,  
 everybody, spite, conscious, least, two}. 
Of POS bi-grams, 13 out of 20 are common to both authors:
NN.CC,  CC.VB,  CC.DT, PP.RB, CC.JJ, IN.WDT, RB.CC, CC.CD,
  TO.WDT, CC.VBD, DT.NNS, NNS.CC, RBR.RB and 
 similarly, they share 13 of of the 20 highest rated POS
 tri-grams:  DT.NN.CC, NN.IN.WDT, CC.VB.DT, IN.WDT.PP, NNS.RB.CC,TO.PP.IN,
 NN.RB.CC, NN.CC.VB, NNS.CC.RB, DT.NNS.CC, RB.DT.NNS, NNS.CC.VB,
 CC.DT.NNS. 
 Moreover, for word uni-grams and POS bi-grams, the first three features 
 occur even in the same order. 
 
 The high number of shared items for all three feature types means
 that case 1 of Distinctiveness prevails in the comparison, meaning that values for shared
 features occur with a very different frequency for each author.
 
 This is consistent with the clustering results shown in table~\ref{tab:results-clust},
 where for all iterations of cross-validation, the result corresponds to
 the ideal clustering result. 
 
 
Figure~\ref{fig:tj-unigram-mdsplot} and figure~\ref{fig:tj-unigram-dendo}
show the MDS plot and dendrogram corresponding to the shared word uni-gram 
features of the first iteration (\emph{J-c}). 
In both one can observe the formation of distinct clusters for both authors 
with no misclassified document. 
 
\begin{figure}[h!]
\centering
\parbox{8cm}{
\includegraphics[width=8cm,height=4cm]{mdsplot/tj-wuni-J-c.pdf}
\caption{MDS plot on James and Twain highest shared word uni-grams 
corresponding to iteration 1.}
\label{fig:tj-unigram-mdsplot}}
\qquad
\begin{minipage}{6cm}
\includegraphics[width=6cm]{dend/tj-wuni-cl-Jc.pdf}
\caption{Dendrogram on James and Twain highest shared word uni-grams 
corresponding to iteration 1}
\label{fig:tj-unigram-dendo}
\end{minipage}
\end{figure}

\subsection{Discussion}
The comparison for  both author sets yielded highly distinctive features
that were well able to separate the two authors in each study. 
The second experiment opposing Henry James and Mark Twain was 
even more successful than the one comparing Charles Dickens and Wilkie Collins.
There might be various reasons for this difference in accuracy. 
In terms of stylistics, James and Twain are likely to vary more
in style (and topic) than Dickens and Collins. 
They also wrote for different audiences, while James' books probably
appealed mostly to the upper classes, Twain enjoyed general 
popularity. They also did not approve of each other 's work, which
is less likely for Dickens and Collins, who collaborated
on at least one work. 
Additionally, one might consider this supported by the fact that 
while Dickens and Collins were best separated on word uni-grams, 
with syntactic structures being less well in separation, which 
might be interpreted as a more intrinsic similarity of building
text than James and Twain, for which any of the three feature types
returned ideal separation results. 

However, one has to consider possible outside influences 
onto the two data sets, such as the prosaic factor of data set size
and the time difference. 
Representativeness and Distinctiveness normalises for the number of 
comparisons and differences in set size between the compared authors.
Yet, there will be a difference between small and bigger sets of an 
author with respect to a particular feature. If the set is small, 
it is more probable that values over different documents of 
an author \emph{agree} for a particular feature, whereas the 
larger a set becomes, the more different documents have to be 
\emph{satisfied}. 
Thus, the slight difference in data set size 54 compared to 46
might have influenced the favourable result for the James/Twain
comparison. 
The other less tangible difference between the sets is time and
cultural differences in terms of 'Britishness' vs 'Americanness'.

Figure~\ref{chron:lifeline} shows the lifespan of all four authors
with a considerable overlap of all four writers. 
If there is a difference in the way stylistic change happened 
for British and American writers at the time, since
we select consistent features over their whole life's works, 
if somehow the two Americans changed considerably \emph{more}
over time, this should rather introduce less consistency in 
their own works and more difficulty in separating them from others. 


% \begin{itemize}
%  \item Dickens: 1812 -- 1870 
%  \item Collins: 1824 -- 1889
%  \item Twain: 1835 -- 1910
%  \item James: 1843 -- 1916
% \end{itemize}



\begin{figure}[h!]
\small
\caption{Lifelines of Dickens and Collins /Twain and James in parallel.}
\label{chron:lifeline}
 \begin{chronology}[10]{1812}{1920}{15cm}{10cm}

% early works
%\event[1812]{1870}{\color{blue} }
\event{1812}{birth: Dickens}
\event{1870}{death: Dickens}

\event{1824}{birth: Collins}
\event{1889}{death: Collins}

\end{chronology}

 
 
\begin{chronology}[10]{1812}{1920}{15cm}{10cm}


\event{1835}{birth: Twain}
\event{1910}{death: Twain}

\event{1843}{birth: James}
\event{1916}{death: James}

\end{chronology}

 \end{figure}









\begin{table}[h!]
\small 

\caption{Results of Adjusted Rand Index for feature types of word uni-grams, POS bi-grams and POS tri-grams.}
\label{tab:results-clust}
 \begin{tabular}{lrrrlrrr} \toprule[1.2pt]
  \multicolumn{4}{c}{\textbf{Dickens vs. Collins}} & \multicolumn{4}{c}{\textbf{James vs. Twain}} \\
   \cmidrule(r){1-4} 
    \cmidrule(r){5-8} 
Test Doc (D./C.) & word unig. & POS bigr. & POS trigr. & Test Doc (J./T.) & word unig. & POS bigr. & POS trigr.\\ \midrule
D1023    & 0.93 & 0.79 & 0.85 &        J-c    &    1    &    1    &    1   \\
D1400    & 0.93 & 0.66 & 0.93 &      J-itc    &    1    &    1    &    1   \\
D19337   & 0.93 & 0.79 & 0.85 &       J-pc    &    1    &    1    &    1   \\
D40723   & 0.93 & 0.79 & 0.85 &     J-poal    &    1    &    1    &    1   \\
D564     & 0.93 & 0.79 & 0.93 &       J-rh    &    1    &    1    &    1   \\
D580     & 0.93 & 0.79 & 0.85 &       J-ta    &    1    &    1    &    1   \\
D588     & 0.93 & 0.79 & 0.85 &      J-taa    &    1    &    1    &    1   \\
D650     & 0.93 & 0.85 & 0.93 &     J-tamb    &    1    &    1    &    1   \\
D675     & 0.93 & 0.66 & 0.79 &      J-tap    &    1    &    1    &    1   \\
D699     & 0.93 & 0.85 & 0.85 &       J-tb    &    1    &    1    &    1   \\
D700     & 0.93 & 0.79 & 0.85 &       J-te    &    1    &    1    &    1   \\
D730     & 0.93 & 0.79 & 0.85 &      J-tgb    &    1    &    1    &    1   \\
D766     & 0.93 & 0.79 & 0.85 &      J-tit    &    1    &    1    &    1   \\
D821     & 0.93 & 0.85 & 0.85 &       J-to    &    1    &    1    &    1   \\
D872     & 0.93 & 0.79 & 0.79 &      J-toh    &    1    &    1    &    1   \\
D882     & 0.93 & 0.85 & 0.85 &     J-tots    &    1    &    1    &    1   \\
D883     & 0.93 & 0.79 & 0.85 &       J-tr    &    1    &    1    &    1   \\
D914     & 0.93 & 0.85 & 0.93 &      J-tsf    &    1    &    1    &    1   \\
D916     & 0.93 & 0.79 & 0.79 &     J-tsop    &    1    &    1    &    1   \\
D917     & 0.93 & 0.79 & 0.85 &    J-tsotp    &    1    &    1    &    1   \\
D963     & 0.93 & 0.79 & 0.85 &      J-ttm    &    1    &    1    &    1   \\
D967     & 0.93 & 0.79 & 0.85 &    J-twotd    &    1    &    1    &    1   \\
D968     & 0.93 & 0.79 & 0.85 &      J-waw    &    1    &    1    &    1   \\
D98      & 0.93 & 0.66 & 0.85 &      J-wmk    &    1    &    1    &    1   \\
Dal2324  & 0.93 & 0.79 & 0.85 &       J-ws    &    1    &    1    &    1   \\
Dal28198 & 0.93 & 0.85 & 0.85 &  T-acyikac    &    1    &    1    &    1   \\
DC1423   & 0.93 & 0.79 & 0.79 &    T-adbds    &    1    &    1    &    1   \\
C1438    & 0.85 & 0.85 & 0.85 &      T-ata    &    1    &    1    &    1   \\
C155     & 0.93 & 0.79 & 0.93 &     T-cfma    &    1    &    1    &    1   \\
C1586    & 0.93 & 0.79 & 0.85 &       T-cs    &    1    &    1    &    1   \\
C1587    & 0.93 & 0.66 & 0.85 & T-fteajatw    &    1    &    1    &    1   \\
C1588    & 0.93 & 0.85 & 0.85 &       T-ia    &    1    &    1    &    1   \\
C1621    & 0.93 & 0.93 & 0.85 &     T-lotm    &    1    &    1    &    1   \\
C1622    & 0.93 & 0.79 & 0.85 &    T-proja    &    1    &    1    &    1   \\
C1623    & 0.93 & 0.85 & 0.93 &       T-ri    &    1    &    1    &    1   \\
C1624    & 0.85 & 0.85 & 0.85 &     T-snao    &    1    &    1    &    1   \\
C1626    & 0.93 & 0.79 & 0.72 &      T-tac    &    1    &    1    &    1   \\
C1627    & 0.93 & 0.79 & 0.85 &    T-taohf    &    1    &    1    &    1   \\
C1628    & 0.93 & 0.79 & 0.85 &    T-taots    &    1    &    1    &    1   \\
C1629    & 0.93 & 0.85 & 0.85 &      T-tet    &    1    &    1    &    1   \\
C1630    & 0.93 & 0.79 & 0.85 &  T-tgaatot    &    1    &    1    &    1   \\
C170     & 0.93 & 0.85 & 0.85 &      T-tms    &    1    &    1    &    1   \\
C1895    & 0.93 & 0.85 & 0.85 &    T-tpatp    &    1    &    1    &    1   \\
C1917    & 0.93 & 0.85 & 0.79 &      T-tsa    &    1    &    1    &    1   \\
C1975    & 0.93 & 0.79 & 0.79 &      T-tsd    &    1    &    1    &    1   \\
C28367   & 0.93 & 0.85 & 0.93 &    T-ttopw    &    1    &    1    &    1   \\
C3606    & 0.93 & 0.79 & 0.79 &             &         &         & 		\\
C3632    & 0.93 & 0.79 & 0.85 &             &         &         & 		\\
C3633    & 0.93 & 0.79 & 0.79 &             &         &         & 		\\
C4605    & 0.93 & 0.85 & 0.85 &             &         &         & 		\\
C583     & 0.93 & 0.66 & 0.79 &             &         &         & 		\\
C7891    & 0.85 & 0.85 & 0.79 &             &         &         & 		\\
C7893    & 0.93 & 0.79 & 0.79 &             &         &         & 		\\
C7894    & 0.93 & 0.79 & 0.85 &             &         &         & 		\\
\bottomrule
\end{tabular}
\end{table}




\section{Conclusion}\label{sec:conclusion}



%\nocite{Sag} % items in your bibliography file that are not cited in your text


\bibliographystyle{clin} 
\bibliography{biblio}  



\end{document}
