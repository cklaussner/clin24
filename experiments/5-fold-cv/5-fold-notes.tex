\documentclass[a4paper]{article}
\usepackage{ifxetex}
\ifxetex
 \usepackage[no-math]{fontspec}
 \defaultfontfeatures{Ligatures=TeX}
 \setmainfont{Times New Roman}
\else
 \usepackage{times}
 \usepackage[utf8]{inputenc}
\fi
\usepackage[american]{babel}

\usepackage{euler}
\usepackage{xcolor}
\usepackage{booktabs}

%\usepackage{tikzstuff}
\usepackage{graphicx}

\newcommand{\TITLE}{R\&D for author profiles: notes on clustering evaluation}
\usepackage{hyperref}
\hypersetup{
colorlinks=true,
linkcolor=[rgb]{0 0 0.2},
urlcolor=[rgb]{0 0 0.2},
filecolor=[rgb]{0 0 0.2},
citecolor=[rgb]{0 0 0.2},
pdftitle={\TITLE},
pdfauthor={Çağrı Çöltekin}
%pdfauthor={Ça\u{g}r\i{} Çöltekin}
}

\usepackage[style=authoryear,backend=biber]{biblatex}
%
% comment out the above, and uncomment the following for apa-like
% citations. Note the dependence on babel above.
%
%\usepackage[style=apa,backend=biber]{biblatex}
%\DeclareLanguageMapping{american}{american-apa}

\addbibresource{\jobname.bib}

\title{\TITLE}
%\author{Çağrı Çöltekin}

\newcommand{\rd}{R\&D{}}

\begin{document}
\maketitle
\pdfbookmark[0]{\TITLE}{\TITLE}

The following are informal notes from some of experiments with the authorship data we are working on.

Our aim is to find representative (R) and distinctive (D) items/features for an author.
Short of a gold standard that defines what is a representative feature for a(ny) author,
we resort for evaluating these items with clustering.
The idea is that if a feature is really \rd{} for an author,
it should be a good feature for identifying the authors.

In the two-author setting we use so far,
our approach so far has been to find \rd{} features for both authors
(the author profiles) separately,
and pick common elements from both profiles for cluster evaluation.
The sizes of the profiles were chosen based on a threshold
(about twice the mean \rd{} value).
And the criteria for choosing the best terms/features were motivated by linearly combined \rd{} values.

In what follows, we will go through a couple of experiments investigating the following:

\begin{itemize}
    \item profile size
    \item the separation ability of \rd{}-based profiles
    \item the effect of combined score
\end{itemize}

\section*{Profile size}

We expect the \rd{} score to be higher for terms or (hopefully) syntactics constructions used by the author distinctively and consistently.
It would be interesting to know how many of these items are usable (this is in a way the same question we've been asking with Jelena in dialect discrimination).

The problem is rather complicated.
As in any setting with dependent items, 
discriminative power of two highly discriminative items may not just add up.
Different number of items may be more suitable in different settings.
In any case, we expect more items to be useful (allowing better discrimination/attribution),
but too many would be confusing for the clustering algorithms.

\includegraphics[width=\linewidth]{DC-wuni-rfl-psize-ri-1-10}\\
\includegraphics[width=\linewidth]{DC-posbi-rfl-psize-ri-1-10}\\
\includegraphics[width=\linewidth]{DC-postri-rfl-psize-ri-1-10}\\
\includegraphics[width=\linewidth]{TJ-wuni-rfl-5000-psize-ri-1-10}\\
\includegraphics[width=\linewidth]{TJ-posbi-rfl-psize-ri-1-10}\\
\includegraphics[width=\linewidth]{TJ-postri-rfl-5000-psize-ri-1-10}\\



\nocite{prokic2012}
\printbibliography
\end{document}
